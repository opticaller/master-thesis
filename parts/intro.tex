\chapter {Introduction}

Software extensions or plugins are used to add additional or optional
capabilities to software. These plugins are often be written by third-parties,
and not the software developers themselves. Such code can potentially be very
malicious and wreak havoc on an user's system when running unrestricted code. It
is not uncommon that a plugin runs native code with the same privileges as the
hosting application. [can I get a quote?]

Very often, this is not desired. Most plugins only need to be able to access
a few functions in a certain API, in order to communicate back to the process.
They might not need to write files to disk, or read an user's private data. This
is where sandboxing comes in. Sandboxing is a security mechanism for running
untrusted programs (or plugins) safely. A good sandbox provides a restricted set
of resources that the code is allowed to use.

Following the principle of least privilige[?], plugins should only be allowed to
use as much, or as little, that they actually require to perform their tasks.
An example where where programs explicitly request access to resources are
Android applications, where at least a few resources, such as network access
have to be explicitly requested upon installation. The user can then decide
whether they trust the program to use those resources or not.

A few solutions to the sandboxing problem involve either limiting the plugin to
use a scripting language using a specific API, or running each plugin in a
separate process, and using operating-system specific sandboxing features for
lowering their priviliges and restricting their potential impact.

Communication between processes and processes can be costly and having separate
processes introduce overhead.
<FORTSÄTT HÄR>
IPC (Inter-Process Communication) techniques used by a process-based extension
to communicate with its host back and forth can be costly.

Thread-driven extensions within the host process can be driven by a thread
pool, potentially reducing the number of context switches significantly.

Processes NOT FAST ENOUGH.

TODO: Mention that the extensions are C/C++, running native code.
