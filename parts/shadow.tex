\chapter {Shadow Memory}

The shadow-memory model used by AddressSanitizer is very simple.
A process' entire memory space gets mapped to a fraction of it.
For instance, with an $1:8$ compression, one continuous eigth of the addressable
memory space gets reserved.
$8$ bytes of memory in the address space then correspond to one byte in this
map.
The value of the byte encodes which of the $8$ bytes are allocated.
We'd like to point out that for this thesis, the compression ratio between
shadow memory and main memory is a power of two.
Though it maybe technically possible to use other alignments, there should be no
sane reason to explore this area, LLVM doesn't support those alignments and
instrumented code can be more efficient with powers of two.

AddressSanitizer instruments LLVM Bitcode so that, when allocating memory, the
corresponding fraction of addresses get marked as allocated in the shadow
memory.
The code is also instrumented so that, before every instruction which operates
on memory, assertions are inserted to check their correctness before actually
performing them.

We propose a different encoding into the same shadow-map scheme.
Whereas AddressSanitizer encodes into this byte how many of the corresponding
$n$ ($n = 8$ in the previous case) original bytes have been allocated, we
propose encoding a plugin-instance id instead.
This way we lose granularity and no longer verify memory usage on a byte level,
as a single $n$-byte block is either ok to read or write, or not.
It would have been possible to fit both a plugin id along with
AddressSanitizer's original encoding, as they effectively use only a few bits of
the shadow bytes (for $n=8$)[ref-asan].
This would have given us fewer plugin ids, and a slower and larger instrumented
check, but kept byte precision.
The proposition is primarily a means of protection and isolation, not a
debugging tool.
But in the context of debugging, the option to compile a plugin with this check
kept could be a good idea.

For shadow-map specfics, we would like to point out AddressSantizer as the
authoritive implementation of this shadow-memory technique. We have been able to
reproduce the essential memory mapping in AddressSanitizer, with simliar
overhead, but their implementation in Clang is way more mature. Their
paper[ref] also provide a good benchmark over performance hits imposed by the
technique.


\section {Whitelist Technique}

AddressSanitizer is primarily a tool used to detect out-of-bounds accesses to
allocated memory.
As such, instead of whitelisting allocated memory, it allocates additional
memory to insert a so-called red-zone on both sides of the allocated memory.
The red-zone is blacklisted, or poisoned, in the shadow memory.
For AddressSanitizer, the default shadow-memory value of $0$ means that a small
chunk of memory is entirely accessible.

Though this allows for a constant $O(1)$ marking of allocated memory, regardless
of how large a heap-allocated chunk may be, it is unfortunately not appropriate
for our case.
Plugins need to be isolated from accessing memory which is owned by
uninstrumented code, yet still allocated by the process.
The default state for shadow memory should therefore be `poisoned'.
As such, we're able to protect against all memory accesses to memory which has
not been explicitly requested by the plugin instance itself.
Inherently, we don't need to insert redzones between variables either.
This could be done as a further error-detection feature, but it's not required
for isolation purposes.


\section {Shadow Compression}

A single-level shadow map makes a $1:1$ byte-to-byte mapping impossible, the
shadow memory would occupy the entire memory space and leave no room even for
the running code.
$1:2$ mappings would reserve half of the address space and so forth.
On 64-bit systems this is not a major issue, as the address space are many
orders of magnitude larger than the physical memory available.
32-bit systems however have significantly limited address space (4GB, a $1:2$
mapping would occupy 2GB).

The compression ratio between real memory and shadow memory directly affects
performance.
If an aligned memory access is smaller than the compression ratio then only a
single check in the shadow is needed.
This is crucial to be able to provide access to basic types, such as
\texttt{int}, \texttt{float} and \texttt{double}, with acceptable overhead.


\section {Memory Management}

Any memory usable by the plugin directly (without external function calls) needs
to, regardless of how it was allocated, be marked as owned by the plugin
instance before it can being used.
After the memory is no longer accessible to the plugin instance it needs to be
unmarked.

\subsection {Memory Alignment}

[TODO: maybe figure here, sliding window for access size]

As memory-access is granted based on memory blocks and not individual bytes,
memory must be handled in multiples of the compression ratio and be aligned to
the compression ratio boundary.
For a $1:8$ ratio, that means all memory allocation must occur in multiples of
$8$ and its base address be a multiple of $8$.
This should, of course, be invisible to the plugin and sufficient memory should
be allocated to accomodate the requested allocation size aligned to the proper
boundary, as well as allocating any remaining space in the final block as all of
it will be marked as accessible.

There is however a boundary case not accounted for in the paragraph above.
When memory is accessed, there exists no guarantee for memory pointers to be
aligned to their access size.
A memory access within a block aligned to its access size rounded up to the
nearest power of two will stay inside the block.
A potential problem arises when an unaligned access at the end of a memory block
occurs.

Memory accesses not aligned to this boundary can easily occur.
This includes maliciously by introducing an unaligned pointer and accidentally
through incorrect pointer manipulation and typecasts.
Whenever such an access is done, it may or trigger a segmentation fault on
certain platforms, depending on the actual alignment.
This is caught by the host environment.
But whenever it falls through, it could overwrite data not owned by the plugin
instance itself.
We prefer to avoid relying on such hosting-environment-specific behavior.

Our proposed solution is primarily a means of protecting the host and other
plugins and not an error-detection tool primarily.
Therefore having such a case go undetected is deemed ok, so long as it doesn't
cause any damage.
We're already missing some erroreous boundary accesses from always allocating a
certain multiple of bytes.
We therefore propose that, whenever memory for $n$ blocks should be allocated,
regardless of how, we allocate an additional garbage block. Thus we allocate
$n+1$ blocks, but only mark the shadow corresponding to the legitimate first $n$
blocks.
This way accesses spilling past the final block will only be able to affect the
garbage block as well, and not alter any outside data belonging to, for
instance, the memory allocator.
The plugin instance will have behaved incorrectly, but not in a way that has
altered any external data.

\subsection {Heap Allocation}

Heap-allocated memory is a resource which needs to be tracked and validated when
sent as a parameter to \texttt{free()} or \texttt{realloc()} etc.
A plugin instance should not be able to free random memory addresses, as this
can cause serious memory corruption.
As the host needs to shut down a plugin instance gracefully it needs to know of
which heap-memory segments were allocated to that instance, so they can be
safely returned to avoid memory leaks.
The host also needs to keep track of how large the memory segments are, so that
the corresponding call to free the memory can also unmark the corresponding
shadow memory.
This can be done by wrapping existing memory-allocation and memory-deallocation
calls and does not require a custom memory allocator.

An easy way to track these resources is to have a hash table for each plugin
instance which maps the allocation's base address to its size.
If there space is empty when a request to free this memory segment is performed,
then the memory base address wasn't allocated by the plugin instance and the
plugin instance should be aborted.
When a plugin instance is shut down, then iterate through the hash table and
free and unmark all still-allocated addresses.
Optionally a warning concerning the memory leak could be issued as well.

\subsection {Stack Allocation}

TODO: Mark: Stack allocation, local variables
TODO: Unmark: End of scope

Stack allocation is a bit more tricky than heap allocation.
With heap allocation every allocation and deallocation is explicit. Stack
allocation however is not.
To free stack memory, what's essentially done is moving the stack pointer.
Anything past the stack pointer is considered garbage.
This will however not automatically mark everything as freed in the shadow
memory, and previously-allocated memory would still be marked as accessible,
which would be incorrect.

At first glance, a simple solution to the stack-allocation problem would be to,
once and for all, mark each entire stack space upon thread creation to belong to
a plugin instance.
Apart from forcing a $1:1$ mapping between plugin instances and threads, this
has serious security implications as well.
The plugin instance should only be able to access the memory on the stack used
for variables.
It should not be able to otherwise alter the execution stack in any manner.

If such alteration were possible by a plugin instance it could, intentionally or
not, introduce buffer overruns and quite easily smash the stack[ref-phrack] to
break execution flow.
This would potentially grant the plugin instance to break out to shell and
execute arbitrary commands.
We'd rather make the potentially over-cautious assumption that not properly
preventing writes to parts of the stack not allocated for local variables or
spilled registers will allow uninstrumented arbitrary-code execution.
Regardless of the bug's potential impact, a plugin instance would preferably
shut down when attempting to perform it.

TODO: How to do alloca? Replacing with malloc wouldn't help with complexity of
the solution. Would still have to free() after.

Execution flow which deviates from normal stack-based subroutine behavior will
be handled in Chapter [chap-exception].
The same principle applies though, stack memory should be marked as owned by the
plugin instance when and only when can potentially be accessed by the plugin
instance.
