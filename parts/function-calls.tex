\chapter {Function Calls}

The ability to call external function is crucial to communicate with the host in
a simple manner.


\section {External Functions}

Apart from utility functions like algorithms and utility functions external
function calls constitutes an ability for the plugin to communicate with its
host. Without any external function calls (or inline assembly of these
functions, which we strongly suggest forbidding in Chapter[ref-instr]) no
communication back to the host environment can be initiated by the plugin.
Though plugins could technically be driven entirely with callbacks, unless the
plugins are limited to performing very specific tasks, we find this highly
impractical.

Potential for different sets of functions, depending on what the user or other
settings have specified. Similar to Android's access levels. In some way.

\subsection {Whitelist Policy}

External function calls need to be allowed or disallowed based on a whitelist
policy.

Principle of least privilige. (Does not apply to pure functions like sin())

The hosting environment can't 

\subsection {Host Extension API}

\subsection {Standard Libraries}


\section {Function Pointers}


