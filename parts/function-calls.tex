\chapter {Function Calls}

Subroutines, or functions, are an important part of the structured-programming
paradigm, and they are centric to most modern programming languages. Without
enforcing great restrictions, a plugin will require calling other functions.

Most internal function calls should be safe. The plugin is not allowed to
rewrite its own code, so any call which explicitly calls a function residing
inside the plugin will finally be compiled to correctly calling instrumented
code which is considered safe. Incorrect indirect function calls could be
generated by calling incorrect function pointers. This could allow the plugin to
at least theoretically call functions which are linked to the program, but
shouldn't be accessible to the plugin. It could also result in the plugin
jumping to something which isn't a start location of a function and then attempt
to run arbitrary garbage. We also suspect that it could possibly allow the
plugin to bypass parts of the instrumented code by entering the function just
after validity checks should have been performed, but before the actual
instruction was performed.

External functions reside outside the protection domain, and running arbitrary
external functions with arbitrary parameters could very possibly allow the
plugin to execute a shell with the user's privileges and escalate its privileges
to run arbitrary code outside its protection domain. The ability to call
external function calls is however crucial to communicate with the host in a
simple manner.


\section {External Functions}

Apart from utility functions like algorithms and utility functions external
function calls constitutes an ability for the plugin to communicate with its
host. Without any external function calls (or inline assembly of these
functions, which we strongly suggest forbidding in Chapter[ref-instr]) no
communication back to the host environment can be initiated by the plugin.
Though plugins could technically be driven entirely with callbacks, unless the
plugins are limited to performing very specific tasks, we find this highly
impractical.

Potential for different sets of functions, depending on what the user or other
settings have specified. Similar to Android's access levels. In some way.

\subsection {Whitelist Policy}

External function calls need to be explicitly allowed based on a whitelist
policy. Blacklisting all functions which can not be 

Principle of least privilige. (Does not apply to pure functions like sin())

The hosting environment can't 

\subsection {Host Extension API}

\subsection {Standard Libraries}


\section {Function Pointers}


