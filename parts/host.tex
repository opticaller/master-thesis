\chapter {Hosting Environment}

The host is responsible for providing a state in which the plugin can be loaded,
unloaded and execute properly, a hosting environment.
This of course includes loading/unloading, restarting and shutdown of plugins,
as well as initialization of any data that may be used by the plugins themselves
etc.
Though the compilation pass used to instrument the code should ideally be done
by the host itself, as incoming code should be untrusted, this policy along with
implications, along with the pass itself is described later in Chapter [ref].

The host is also responsible for keeping track of plugin resources and providing
an API to allocate, access and free these.
This includes usage of standard memory-allocation functions such as
\texttt{malloc} and \texttt{new}, keeping track of file descriptors etc.
Associating these resources with a specific plugin instance prevents resource
leaking after the instance has been shut down.

Even though instrumented plugin code is considered \emph{trusted}, certain
operations, such as integer divisions aren't checked.
Thus it's still possible for plugins to raise signals in some environments for
certain operations, such as division by zero.
Though this particular case could have been dealt with by instrumenting every
single divide instruction, it's easier and more effective to have a signal
handler to catch these types of errors, which trigger hardware interrupts and
have no additional overhead when used correctly.
If the error triggers a more severe error than a signal which can be caught, an
instrumented check is required. Apart from raising these signals which can be
caught by the host, the instrumentation pass is responsible for generating
instrumented code that aborts before (if) any code which would have behaved
maliciously or otherwise incorrect executes.

The hosting environment includes external functions available to the plugin,
including, not only the host's plugin API but also library functions in the C
standard library and any system-specific functions such as the POSIX API.
For instance, if a plugin raises the POSIX signal \texttt{SIGKILL}, the signal
can not be blocked or ignored by the program, forcing it to abort[ref]. External
functions, including plugin APIs are covered further in Chapter [ref].

TODO: More on this chapter after performing actual implementation..

\section {Plugin Types}

TODO: Plugins can take many forms, blah blah.

\subsection {Continuously-Running Plugins}

A plugin which continuously runs, occasionally requesting/providing information
back to the host naturally occupies at least one thread.
Though a multi-threaded plugin would also be possible, the checks performed by
a thread and the actual instruction are not performed atomically together.
Memory could be shared as normal if all threads are assigned the same
plugin-instance ids, but the reader should be aware of the race conditions which
this model could introduce.

The model proposes that the instrumented code performs each memory operation in
two steps: Validate a memory operation, then perform it.
Since this isn't done atomically, another thread running the same plugin
instance could invalidate the check before, or even during, the actual memory
operation.
For instance, freeing a previously allocated block when the other thread writes
to it.
Another plugin could even have allocated and rewritten the entire block.
To be consistent this two-step check has to be performed atomically.
The trivial solution would be to add a shared global lock for every single
memory operation, which would have devastating consequences for performance.

One other potential solution is to have something like `checkpoints' where all
threads wait, and when all running threads are either trying to allocate memory
or waiting at checkpoints any memory allocations are performed and the threads
could then resume execution.
This would likely impose significant overhead.

For a one-to-one mapping between threads and plugins like these (single-threaded
continuously-running plugins), we expect nothing out of the ordinary.

\subsection {Event-Based}

Plugins respond to events, such as audio-buffer callbacks. This can take the
form of tasks, where §

TODO: Relate to Rack Extensions.

\subsection {Hybrid Model}

TODO: Both continuously running and callbacks.


\section {Resource Limits}

TODO: Blah blah, too much memory, using too much cpu, not responding fast enough
(tick takes forever).


\section {Global Initialization}

Init shadow mem etc.


\section {Plugin Startup}

How load plugin? How load new instances. What are? dlls. etc?


\section {Plugin Switching}

Thread-pool stuff.


\section {Error Recovery}

Ideally a plugin instance should shut down gracefully when misbehaving.
Allocated resources, such as memory or file descriptors, should be deallocated
or closed.
These resources need to be tracked, even for the case where a plugin doesn't
per-se misbehave at any point of execution.
Even if a plugin exits properly, but haven't for instance closed file
descriptors or freed all heap-allocated memory, there would be resource/memory
leaks, potentially quite severe.

The requirements for properly sandboxed plugins are more strict than that
however.
A malicious plugin can call any external functions with completely arbitrary
arguments. This includes double-freeing heap-allocated memory, a bug which can
lead to serious memory corruption for the entire process from which it is not
able to recover.

This provides a clear protection domain.
Any function calls external to the module must be individually examined, and all
arguments sanity checked.
\texttt{div} for instance can't expect a non-zero divisor as an argument.
It also means that no external function call can expect a following call with
certain arguments; \texttt{malloc} can't expect a matching \texttt{free},
\texttt{fopen} can't expect a matching \texttt{fclose}. External functions will
be discussed further in Chapter {chapref}. 

In short, there should be no state from which the host can't recover. No code
should, after being instrumented, be able to make the host environment
misbehave.

\subsection {Signal Handling}

Signals indicating errors should cause a plugin instance to shut down, and
indicate to the managing thread that this occurred. The host can then free
resources allocated to the plug-in. The host can then optionally reboot the
thread and plugin.

TODO: Describe that signal handler.
