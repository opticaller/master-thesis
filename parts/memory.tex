\chapter {Memory Isolation}

To protect the host and other plugins from having their heaps and/or stacks
modified, memory accesses between plugins and the host (and other programs) must
be checked in one way or another. Memory protection is a part of most modern
operating systems, and often has hardware support. This mechanism protects other
processes from incorrect memory accesses to their memory. Naturally, all memory
within a single process is shared by every thread of execution and all access
control implemented on a per-process process level [quote fuckyes?].
 
Inherently, this built-in protection offered by the operating system is not
applicable to a plugin model where memory is shared, such as in the thread-based
environment provided by Reason.

TODO: Memory-error checking

TODO: Shadow memory

The extension should be able to access its stack variables and dynamically
allocated memory.

\section {Dynamic-Memory Allocation Model}

\subsection {Shared Heap}

\subsection {Per-Extension Address Spaces}

Placing each extension's stack and heap into its own continuous address space
allows for a single coarse-grained check against start and end address. On
32-bit platforms virtual address space is limited however, and if there's no
good estimate on how much memory an extension will use, the extension might
either have too much space allocated and the program's total address space
might run out when using too many extensions.

On the other hand, if the address space for an extension runs out, it can't
allocate more memory. Moving an extension's address space would however require
translating every pointer pointing to the block itself as well.

\section {Bounds-Checking Memory Accesses}

Mention Valgrind. Also Valgrind's overhead. Mention ASan.

\subsection {Bounds-Checking Stack and All Blocks From \texttt{malloc()} }

\subsection {Coarse Extension Stack/Heap Bounds}

Provided we place it inside an aligned block of size $2^n$ gives us an address
range similar to \texttt{0x12340000-0x1234FFFF} (on a 32-bit system). To check
whether a byte is within this range, we can simply \texttt{AND} with a bitmask
matching the base-address bits, \texttt{0xFFFC0000}, and check that the result
matches the base address \texttt{0x12340000}.

\subsection {Address Sanitizer}

Stuff about asan.

\section {Tracking Pointer Use}

\section {Load/Store Instructions}

\section {alloca}

\section {Hooking \texttt{malloc()}}

\subsection {Hooking LIBC}

LIBC provides callback hooks that are called after malloc and friends occur.

Doesn't work. free can't occur before it's ok, eg. a plugin shouldn't be
allowed to free a chunk allocated by the main thread.

<TODO References>

\subsection {LD\_PRELOAD}

<TODO + TODO References>
