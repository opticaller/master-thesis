\chapter{Memory Isolation}

To protect the host and other threads from trashing their heaps/stacks or
otherwise modifying their internal variables, memory accesses must be checked.
Memory protection available on most <TODO>

The extension should be able to access its stack and dynamically allocated
memory.

\section{Dynamic-Memory Allocation Model}

\subsection{Shared Heap}

\subsection{Per-Extension Address Spaces}

Placing each extension's stack and heap into its own continuous address space
allows for a single coarse-grained check against start and end address. On
32-bit platforms virtual address space is limited however, and if there's no
good estimate on how much memory an extension will use, the extension might
either have too much space allocated and the program's total address space
might run out when using too many extensions.

On the other hand, if the address space for an extension runs out, it can't
allocate more memory. Moving an extension's address space would however require
translating every pointer pointing to the block itself as well.


\section {Bounds-Checking Memory Accesses}

\subsection{Coarse Extension Stack/Heap Bounds}

Provided we place it inside an aligned block of size $2^n$ gives us an address
range similar to \texttt{0x12340000-0x1234FFFF} (on a 32-bit system). To check
whether a byte is within this range, we can simply \texttt{AND} with its base
address \texttt{0x12340000} and check that the result is the same as the base
address.

\subsection{Bounds-Checking Stack and All Blocks From \texttt{malloc()}
	and \texttt{free()}}

\section{Tracking Pointer Use}

\section{Load/Store Instructions}

