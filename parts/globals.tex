\chapter {Global Variables}

The proposed solution implies that every single piece of memory is owned by a
single plugin instance.
Inherently, globals, as every other piece of memory, can not be shared between
instances.
As globals are naturally shared between every thread in a running program, this
constitutes a problem that needs to be solved if global variables are to be
accessible at all.
Though it's possible in some cases to validate accesses of memory back to a
global variable through static analysis, it does not cover their usage arbitrary
pointers.
Partially allowing uses of global variables where rejected uses are not detected
at compile-time could lead to some nasty bugs which are almost impossible to
track down.

In this chapter, we describe a few very different solutions to the problem.
Their applicability is highly context-dependent, the specific solution should be
chosen after taking several factors into account.
For plugin-instance-isolation reasons non-constant globals were already
forbidden in Rack Extensions.
Therefore, due to time constraints, we decided to go with only instrumenting
storing instructions.
There was no sensitive data which needed protection, and Rack-Extension
developers are actually trusted not to do anything malicious on purpose.
Instrumenting only storing instructions also significantly decreased overhead
both on code size and running speed.


\section {Shared Globals}

If memory access is only separated based on their plugin class and not their
assigned plugin-instance ids then the globals problem becomes a bit easier.
Globals have to be marked before they are used.
As with memory allocation another block past the end of usable memory is
required as well.

This can be solved by putting all globals inside a struct and replacing usage of
them with their in-struct counterpart.
Finally, add an additional memory block to the struct and add alignment up to
block size.
Have this struct be aligned to memory block size or largest alignment inside the
struct, whichever is larger.
We would like to note that after this step is done, no transformations which
will alter this struct, such as scalar replacement of aggregates[ref] can be
run.

Note that in LLVM alignment inside non-packed structures are based on what the
underlying code generator expects[ref].
If global variables require different alignment, then putting globals aligned
into such a struct would require further work and investigation.

\subsection {Read-Only Globals}

Only instrumenting store instructions and making globals read-only by policy is
by far the simplest solution to the global-variable problem. All reads are
naturally accepted, and all writes are naturally rejected, as the global
variables are not marked as accessible by anyone.

This policy conflicts with global variables initialized through global
constructors.
In C a global variable can only take a value, whereas in C++ and other languages
they can be initialized through arbitrary function calls which gets executed
before main.

We see two simple solutions to this problem when using read-only globals: Either
trust the global constructors and leave them uninstrumented, which is very
naive in an untrusted environment, it's essentially allowing uninstrumented
arbitrary-code execution, or enforce the policy that globals really can not be
written to.
A third solution would be marking them as accessible during initialization, as
it is only performed once.
In this final case, it's appropriate to put the globals in a struct like
described above in Section [ref-shared-globals].


\section {Instance-Local ``Globals''}

One solution to the globals problem is easy to think of: Give each plugin
instance its own local copy.
If there are no global variables at runtime then there is no problem left to
solve.
This allows every plugin instance to be as isolated as possible while not
enforcing read-only restrictions etc. on global variables.
Out of simplicity, the globals are assumed to be repacked into a struct, which
is somehow set up for the different instances.
This simplifies an explanation of the model, and it's easier to not have the
globals segmented either way.

As arbitrary global values can include pointers to other global variables
copying them to the rest of the plugin instances after they've been initialized
once is not valid in the general case, because the new set of globals may be
pointing back to the old global struct.
It's possible to track properly typed pointers and translate them between blocks
if they are pointing to inside the globals struct.
Though in the more general case it's also valid to store a pointer with a
certain fixed offset from a global variable, which would leave it pointing
outside of existing globals.
It's not covering all cases, it's not invalid to have pointers pointing just
outside the area, or even storing them typecast into non-pointer types such as
\texttt{intptr\_t} in C99.
Furthermore, any pointers pointing to dynamically-allocated memory would still
point to memory allocated by the instance initially initializing globals.

The solution we suggest for having instance-local globals is to instrument the
global constructors of a plugin to operate on an instance-local struct.
These global constructors would then not run on initialization of a plugin, but
rather on initialization of a plugin instance.
There are some corner cases regarding non-pure constructors which will be
discussed here under Section [ref-non-pure].

We would also like to have investigated the possibility of instrumenting the
code to dereference a pointer to an instance-owned globals struct instead.
This way we could have transferred the plugin instance between threads by simply
passing the pointer to the plugin instance's globals.
This type of instrumentation seemed unsupported by the LLVM API at the time of
writing, and the effort was subsequently dropped.

\subsection {Thread-Local Globals}

Making the global-struct thread local is a simple solution.
The struct can however not be moved around, for the same reasons why a struct
cannot simply be copied mentioned above.
So having this struct thread-local would force that no more than a single
instance of a plugin runs on a thread, and that no plugin instance runs
concurrently on multiple threads.
This would be suitable for a $1:1$ mapping between threads and plugin instances.

\subsection {Impure Global Constructors}

When global functions are not pure, there are some corner cases in which
repeatedly-performed initialization would behave badly.
There are cases where impure functions can be run from a globals perspective
without having problems.
\texttt{int x = rand();} is a fine example, even though the plugin instances
receive different values for x, they shouldn't have been expecting any
particular values for x, just something ``random'', which will still apply for
every single instance.
Memory allocated through \texttt{malloc} will also be fine, as every instance is
expecting a memory block of a certain size, and not a particular pointer value.

A constructed example would be expecting the first character of input from a
stream, for instance, stdin: \texttt{int x = std::cin.get();}.
In this case the plugin instances would all receive different characters, they
would all be expecting the first line of input.
This would leave a nasty bug that's hard to track down.

To solve this we propose providing an API in which the plugin instances don't
have access to impure functions with unwanted side effects.
A plugin shouldn't in general have to get the first line of input directly from
stdin.
In the example stdin could have been logged by the host and an API function
provided to receive the first character of stdin directly.
This usage would be less error prone, and all instances would receive what they
were expecting.
