\chapter {Global Variables}

The proposed solution implies that every single piece of memory is owned by a
single plugin instance. Inherently, globals, as every other piece of memory, can
not be shared between instances. As globals are naturally shared between every
thread in a running program, this constitutes a problem that needs to be solved
if global variables are to be accessible at all. Though it's possible in some
cases to track usage of a variable back to 

In this chapter, we describe a few very different solutions to the problem.
Their applicability is highly context-dependent, the specific solution should be
chosen after taking several factors into account. For the initial implementation
done for Rack Extensions, we decided to go with only instrumenting storing
instructions. For plugin-instance-isolation reasons non-constant globals were
already forbidden. There was no sensitive data which needed protection, and
Rack-Extension developers are trusted not to do anything malicious on purpose.
Instrumenting only storing instructions also significantly decreased overhead
both on code size and running speed.


\section {Shared Globals}

If memory access is only separated based on their plugin class and not their
assigned plugin-instance ids then the globals problem becomes a bit easier.
Globals have to be marked before they are used. One problem arises however, as
with memory allocation we require another block past the end of usable memory to
be allocated as well.

This can be solved by putting all globals inside a struct and replacing usage of
them with their in-struct counterpart. Finally, add an additional memory block
to the struct and add alignment up to block size. Have this struct be aligned to
memory block size or largest alignment inside the struct, whichever is larger.
We would like to note that after this step is done, no transformations which
will alter this struct, such as scalar replacement of aggregates[ref] can be
run.

Note that in LLVM alignment inside non-packed structures are based on what the
underlying code generator expects[ref]. If global variables require different
alignment, then putting globals aligned into such a struct would require further
work and investigation. We however expect this case to be fairly uncommon.

\subsection {Read-Only Globals}

Only instrumenting store instructions and making globals read-only by policy is
by far the simplest solution to the global-variable problem. All reads are
naturally accepted, and all writes are naturally rejected, as the global
variables are not marked as accessible by anyone.

This policy conflicts with global variables initialized through global
constructors. In C a global variable can only take a value, whereas in C++ and
other languages they can be initialized through arbitrary function calls which
gets executed before main.

We see two simple solutions to this problem when using read-only globals: Either
trust the global constructors and leave them uninstrumented, which is very
naive, as it's essentially allowing uninstrumented arbitrary-code execution, or
enforce the policy that globals really can not be written to. A third solution
would be marking them as accessible during initialization, as it is only
performed once. In this final case, it's appropriate to put the globals in a
struct like described above in Section [ref-shared-globals].


\section {Instance-Local ``Globals''}

One solution to the globals problem is easy to think of: Give each plugin
instance its own local copy. If there are no global variables at runtime then
there is no problem left to solve. This allows every plugin instance to be as
isolated as possible while not enforcing read-only restrictions etc. on global
variables. Out of simplicity, the globals are assumed to be repacked into a
struct, which is somehow set up for the different instances. This simplifies an
explanation of the model, and it's easier to not have the globals segmented
either way.

As arbitrary global values can include pointers to other global variables
copying them to the rest of the plugin instances after they've been initialized
once is not valid in the general case, because the new set of globals will
potentially be pointing back to the old global struct. It's possible to track
properly typed pointers and translate them between blocks if they are pointing
to inside the globals struct. It's not covering all cases, it's not invalid to
have pointers pointing just outside the area, or even storing them typecast into
non-pointer types such as \texttt{intptr\_t} in C99. Furthermore, any pointers
pointing to memory that has been dynamically allocated would belong to the
previous instance and that pointer would also be invalid to access after
copying.

The solution we suggest for having instance-local globals is to instrument the
global constructors of a plugin to operate on an instance-local struct. These
global constructors would then not run on initialization of a plugin, but rather
on initialization of a plugin instance. There are some corner cases regarding
non-pure constructors which will be discussed here under Section [ref-non-pure].

We would also like to have investigated the possibility of instrumenting the
code to dereference a pointer to a globals struct instead. This way we could
have transferred the plugin instance between threads by simply passing the
pointer to the plugin instance's globals. This type of instrumentation seemed
unsupported by the LLVM API at the time of writing, and the effort was
subsequently dropped.

\subsection {Thread-Local Globals}

Making the global-struct thread local is a simple solution. The struct can
however not be moved around, for the same reasons why a struct cannot simply be
copied mentioned above. So having this struct thread-local would force that no
more than a single instance of a plugin runs on a thread, and that no plugin
instance runs concurrently on multiple threads. This would be suitable for a
$1:1$ mapping between threads and plugin instances.

\subsection {Impure Global Constructors}

When global functions are not pure, there are some corner cases in which a
plugin instance would behave badly. There are cases where impure functions can
be run from a globals perspective without having problems.
\texttt{int x = rand();} is a fine example, even though the plugin instances
receive different values for x, they shouldn't have been expecting any
particular values for x, just something ``random'', which will still apply for
each single instance. Memory allocated through \texttt{malloc} will also be
fine, as every instance is expecting a memory block of a certain size, and not
a particular pointer value.

A constructed example would be expecting the first line of input from a stream,
for instance, stdin: \texttt{std::string x = std::cin.getline();}. In this case
the plugin instances would receive different lines, they would all be expecting
the first line of input. This would give a nasty bug that's hard to track down.

We propose providing an API in which the plugin instances don't have access to
impure functions with unwanted side effects. A plugin shouldn't in general have
to get the first line of input directly from stdin. Instead stdin could be
logged so that requesting the first line of stdin would be both indempotent,
always returning the same thing, as well as more explicit.

