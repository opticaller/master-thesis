\chapter {LLVM Assembly Language}

\begin{quote}
LLVM is a Static Single Assignment (SSA) based representation that provides
type safety, low-level operations, flexibility, and the capability of
representing `all' high-level languages cleanly. It is the common code
representation used throughout all phases of the LLVM compilation strategy.
\end{quote}

\noindent The LLVM compiler infrastructure provides an easy way of manipulating
code in the LLVM intermediate representation (IR) format. Modifying code
already compiled to this intermediate format enables our implementation to
support multiple languages that are already supported by LLVM front-ends rather
than a specific language.

Operating in this RISC-like intermediate format also provides a clear and
simple distinction between instructions performed on memory and arithmetics
performed on temporary values. Consider an instruction set where \texttt{load}
and \texttt{store} are the only instructions which read or modify memory. In
such an instruction set a compiler pass used for protecting memory only have to
consider these two instructions. LLVM IR has a few more instructions which
operate on memory which we have to consider, but they're still relatively few
and easily enumerable.

Quote: \texttt{http://llvm.org/docs/LangRef.html}

RISC: \texttt{http://www.drdobbs.com/architecture-and-design/the-design-of-llvm/240001128}

\section {Modules}

LLVM programs consists of modules. Each module corresponds to a translation
unit of the compiler and can contain functions, global variables and
symbol-table entries. These modules can be linked together using the LLVM
linker. For the purpose of this thesis we consider a plugin to be a single
module. That is, before running our compiler pass, the plugin has to be linked
into a single module.

What is a module? \texttt{http://llvm.org/docs/LangRef.html}

\section {Instructions}

List of instructions. Which have to be instrumented, and how? Which IR version?

Most instructions are concerning register stuff, don't operate on memory. TODO
SSA

\subsection {Modify reads/writes/both?}

Why? Why not? Wuzzaaaaah?
Depends on purpose!

DEFENDS ON PURPOSE!

\subsection {Type-Safety}

TODO: LLVM IR's type safety, how it affects us, and why it's awesome.

\subsection {\texttt{load}}

And why

how should they be handled?

Size of operand etc.?

\subsection {\texttt{store}}

\subsection {\texttt{llvm.memcpy}}

\subsection {\texttt{bah}}

