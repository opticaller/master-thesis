\chapter {LLVM Assembly Language}

\begin{quote}
LLVM is a Static Single Assignment (SSA) based representation that provides
type safety, low-level operations, flexibility, and the capability of
representing `all' high-level languages cleanly. It is the common code
representation used throughout all phases of the LLVM compilation strategy.
\end{quote}

<Quote: \texttt{http://llvm.org/docs/LangRef.html}>\\

\noindent The LLVM compiler infrastructure provides an easy way of manipulating
code in the LLVM intermediate representation (IR) format. Modifying code
already compiled to this intermediate format enables our implementation to
support multiple languages that are already supported by LLVM front-ends rather
than a specific language.

The type-safety of the LLVM Assembly Language prevents accidental bugs
originating from implicit conversion between parameters. Unlike similar formats
in which a register simply holds `a value', all parameters to LLVM instructions
are explicitly typed. It does not mean LLVM isn't able to pass of incorrect type
to instructions or function calls, it's just that they have to be explicitly
typecast to do so. We have experienced that this can prevent creeping errors
from occuring at all.

Operating in this RISC-like intermediate format also provides a clear and
simple distinction between instructions performed on memory and arithmetics
performed on temporary values. Consider an instruction set where \texttt{load}
and \texttt{store} are the only instructions which read or modify memory. In
such an instruction set a compiler pass used for protecting memory only have to
consider these two instructions. LLVM IR has a few more instructions which
operate on memory which we have to consider, but they're still relatively few
and easily enumerable.

<LLVM being RISC-like: \texttt{http://www.drdobbs.com/architecture-and-design/the-design-of-llvm/240001128}>

Keep in mind that any new instructions added to the LLVM Assembly Language might
also read and write to memory, so any instructions added to the language must
also be inspected (or explicitly forbidden). To detect any new instructions that
have to be considered, it's possible to have a compiler pass which explicitly
whitelists instructions, so that any unexpected instructions would cause
compilation to abort. From a security standpoint, whitelisting instructions
explicitly is crucial to ensure that a new instruction doesn't go unnoticed.

We purposefully do not attempt to provide a complete list of instructions and
whether they require instrumentation or not. As a rule of thumb, instructions
which only operate directly on local registers do not require instrumentation as
they're local to the plugin instance running a particular function. Instructions
with side effects outside of local registers need to be further examined.
\texttt{load} is such an instruction, while it takes a local register as input,
it dereferences the value as an address to read from the arbitrary location
determined by its input.


\section {Modules}

\begin{figure}[ht]
\begin{lstlisting}[language=C]
#include <stdio.h>

int main() {
    puts("Hello, World!");
    return 0;
}
\end{lstlisting}

\begin{lstlisting}[language=llvm]
; ModuleID = 'hello.c'
target datalayout = "e-p:64:64:64-i1:8:8-i8:8:8-i16:16:16-i32:32:32-
                     i64:64:64-f32:32:32-f64:64:64-v64:64:64-v128:128:128-
                     a0:0:64-s0:64:64-f80:128:128-n8:16:32:64-S128"
target triple = "x86_64-unknown-linux-gnu"

@.str = private unnamed_addr constant [14 x i8] c"Hello, World!\00", align 1

define i32 @main() nounwind uwtable {
  %1 = tail call i32 @puts(i8* getelementptr inbounds ([14 x i8]* @.str, i64
       0, i64 0)) nounwind
  ret i32 0
}

declare i32 @puts(i8* nocapture) nounwind
\end{lstlisting}
\caption{A ``Hello World'' program compiled by Clang to a LLVM Module}
\end{figure}

\noindent LLVM programs consists of Modules[ref-lang]. A Module normally
corresponds to a translation unit of the compiler. It can contain functions,
external declarations, global variables and symbol-table entries. It also
contains information about the target target compiled to. The datalayout string
contains information about the target, such as endianness, data-type sizes and
alignment. The example target triple tells us that the target is a GNU
environment on Linux running on an x86-64 architecture.

These Modules can be linked together using the LLVM linker into a single Module.
For the purpose of this thesis a plugin will be treated as a single Module. That
is, before running our proposed compiler pass, the plugin has to be linked into
a single Module.

< ref-lang: \texttt{http://llvm.org/docs/LangRef.html} >


\section {Instructions}

An instruction specifies a single operation of an assembly language. They are
the smallest unit of code. In machine code they are the only operations that a
processor can perform directly. To be able to perform other operations several
of these instructions are combined, often into larger subroutines.

Unlike CISC, and like for most RISC-like instruction sets, the LLVM Assembly
Language only have a few instructions which operate on memory addresses. It
provides a tiny interface between memory and registers, Instrumentation isn't
required for the operations that only operate on pseudo-registers and have no
side effects apart from assigning values to the pseudo-registers themselves.
LLVM is responsible for properly allocating enough stack memory for register
spilling and to properly address them as well. Were this not the case, we
couldn't expect LLVM to compile its own assembly format into a working binary at
all.

We do however need to be concerned with explicit memory accesses that aren't
pseudo-registers instead. Explicit memory accesses could recide anywhere in
memory, unlike pseudo-registers, which are either spilled onto the stack or kept
inside actual processor registers.

\subsection {Stack Allocation: \texttt{alloca}}

\begin{figure}[ht]
\begin{lstlisting}[language=llvm]
%ptr = alloca i32                               ; yields {i32*}:ptr
\end{lstlisting}
\caption{Stack allocation of a 32-bit integer in LLVM}
\end{figure}

\noindent Stack allocation in LLVM is done using the \texttt{alloca}
instruction. These instructions are interesting as well, they allocate new
memory which can be read to or written later. As the thesis proposes a solution
using shadow memory to look up whether some memory is accessible or not, stack
allocated memory has to be marked as accessible before it can be used by the
plugin.

Allocating memory on the stack is a technically simple operation, by moving the
\emph{stack pointer}, the stack either grows or shrinks, depending on the
direction which it moves in. Therefore, freeing all stack-allocated variables
from inside a function is done automatically when a function returns, and the
stack pointer is moved to remove the top frame from the stack. Unlike
stack-allocated memory however, the corresponding shadow-map entity is not
automatically removed when a function returns however.

We addressed this by prepending all instructions which cause a function to
return with code to remove the shadow-map entities corresponding to the
stack-allocated memory segments. The procedure for doing this will explained
further in Chapter [ref].

\subsection {\texttt{load} and \texttt{store}}

\begin{figure}[ht]
\begin{lstlisting}[language=llvm]
store i32 3, i32* %ptr                          ; stores 3 to *ptr
%val = load i32* %ptr                           ; yields val = *ptr
\end{lstlisting}
\caption{The LLVM \texttt{load} and \texttt{store} instructions}
\end{figure}

\noindent Reads from and writes to memory are interesting operations. In some
circumstances only write protection is required. This depends entirely on
protection requirements imposed by the host environment, it's important to
account for what kind of data the host protects, and quite possibly also what
other plugins could protect. A plugin which provides a PGP[footnote-about-pgp]
encryption feature to an email client would likely, at some point in time, store
a secret encryption key in memory. Its state should definitely not be readable
by other plugins which would otherwise potentially be able to leak the key.

Instrumenting memory accesses give quite some overhead however. A host which
neither needs to protect its internal state from being read nor provide such
guarantees to plugins inside its environment can choose to instrument only
instructions which write to memory. This way a plugin can potentially read all
data inside the running application, but it can not change any of it. This is
still required to prevent system instability caused by a plugin modifying the
host or another plugin's state.

Rack Extensions are such plugins, no really sensitive data is kept in the host,
and as users want to pack as many instruments as possible into the application,
a large slowdown factor really makes the plugin less usable.

\subsection {Volatility and Atomicity}

Volatility and atomicity are both concurrent-programming concepts which fall
outside the scope of this thesis. They are however very relevant to the subject
itself. Volatile stores can be reordered relative to one another. We have not
explored this behavior. As our instrumented code introduces branching, it is
likely that this reordering constraint doesn't affect the order of approval and
execution of memory accesses.

Atomicity however is important. If an address is sanity checked, it follows that
the following instruction must operate on the same address that the check was
performed against. If the pointer variable can be changed from other locations,
then it might be a good idea to copy the pointer into a local variable to ensure
that, though the address may be old when the read/write is performed, at least
the check and execution is both done on the same original address.

Ideally, the instrumented code would be performed atomically along with say a
write instruction, but unfortunately that has some performance implications.
These factors also have to be taken into hand as well. So long as a plugin
instance owns its own memory and runs in a single thread, the memory will not be
free'd by another running while the plugin instance is using it. That is, it
will not permission to use a chunk of memory between passing a check to permit a
memory access and performing the actual memory access. In this scenario, memory
checks appear to be atomically performed together with the write instruction.
