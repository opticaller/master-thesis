\chapter* {Project Introduction}
\markboth {PROJECT Introduction}{}
\addcontentsline{toc}{chapter}{Project Introduction}

\begin {quote}
    \emph {The following paragraphs were written as an introduction to
           Propellerhead and the master-thesis project by Erik Agsjö and Gustaf
           Taxén.}
\end {quote}

Formed in 1994, Propellerhead (\url {http://www.propellerheads.se/}) is a
privately owned company based in Stockholm, Sweden.
Renowned for its musician-centric approach, Propellerhead has created some of
the world's most innovative music software applications, hardware interfaces and
technology standards.

Musicians, producers and the media have praised \emph{Reason}, \emph{ReCycle}
and \emph{ReBirth} applications for being inspiring, great sounding and of
superior quality.
Technologies such as \emph{ReWire} and the \emph{REX} file format are de-facto
industry standards, implemented in all major music software.
Today, Propellerhead’s products are used all over the world by hundreds of
thousands of professionals and enthusiasts for all kinds of music making.

In the summer of 2012, Propellerhead launched one of the most significant
updates to the Reason software: \textbf {Rack Extensions}.
Rack Extensions are third-party plug-ins for Reason that extends the
capabilities of the Reason host in various ways.
Examples include (virtual) instruments, sound effects processors, and audio
analysis utilities.
All Rack Extensions are sold exclusively in the Propellerhead on-line shop,
\url{http://shop.propellerheads.se/}.

Propellerhead focuses very strongly on quality and robustness and the Rack
Extension concept is no exception.
Once a Rack Extension has been approved by Propellerhead, it will be supported
in Reason indefinitely, and songs created with an old version of a Rack
Extension will continue to work even if the Rack Extension is updated by its
developer.
Each Rack Extension runs in a protected sandbox environment inside Reason, which
ensures that the user does not lose any data if the Rack Extension should crash.
Rack Extensions are also highly portable and the format assumes very little
about the operating environment in which the Reason host is running (Reason is
currently available for Mac OSX and Windows, in both 32- and native 64-bit
versions).

Real-time audio processing is computationally expensive, so Rack Extensions are
written in C++ for efficiency reasons.
However, third-party developers do not submit C++ source code to Propellerhead.
Instead, we use the portable byte code format of the LLVM compiler.
This ensures that the implementation details of the third party code are
protected, while allowing Propellerhead to maintain the submitted code
indefinitely, without having to keep in contact with the original developer.

In summary, Rack Extension plugins are \emph{portable}, \emph{computationally
efficient}, and \emph{robust}.
The Reason host already provides a very strong environment with respect to these
requirements, but there are a number of areas that could be improved.
This Master’s Thesis focuses on some of these areas.
