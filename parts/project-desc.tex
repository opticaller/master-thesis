\chapter* {Project Description}
\markboth {PROJECT DESCRIPTION}{}
\addcontentsline{toc}{chapter}{Project Description}

Software extensions or plugins are used to add additional or optional
capabilities to software. These plugins are often be written by third-parties,
and not the software developers themselves. Such untrusted code can be very
malicious and wreak havoc on an user's system when running unrestricted code. It
is not uncommon that a plugin runs native code with the same privileges as the
hosting application. [can I get a quote? examples?]

Most plugins however only need to be able to access a few very specific
functions in a certain API to perform their work and to communicate back to the
hosting process. They might not need to write files to disk, or read an user's
private data. Following the principle of least privilige\footnote{\emph{``Every
program and every privileged user of the system should operate using the least
amount of privilege necessary to complete the job.''} -- Jerome Saltzer,
\emph{Protection and the control of information sharing in multics}, 1974.},
plugins should only be allowed to use as much, or as little, that they actually
require to perform their tasks. An example where where programs explicitly
request access to resources are Android applications, where at least a few
resources, such as network access have to be explicitly requested upon
installation. The user can then decide whether they trust the program to use
those specific resources or not.

Sandboxing is a security mechanism for running untrusted programs (or plugins)
safely. A good sandbox provides a restricted set of resources that the code is
allowed to access. A few solutions to the sandboxing problem involve either
limiting the plugin to use a scripting language using a specific API, or running
each plugin in a separate process using operating-system specific features for
lowering their priviliges and restricting their potential impact.

Communication between processes and just having separate processes can introduce
both overhead from context switches and latency as well. As Propellerhead's
Reason, as a digital audio workstation, has a natural demand for low latency,
all plugins run inside the host process driven by a thread pool. Rack
Extensions, their plugin technology, are real-time digital synthesizers running
native code to process a large amount of data inside a few-millisecond window
created by a callback whenever an audio driver requires an additional buffer to
be filled. Missing such a callback window also comes with a penalty, such as
crackling sound. Minimizing any overhead is crucial to be able to run as many
instances of these synthesizers as possible.

For such extreme performance requirements, even being at the mercy of the OS
scheduler was not an option. Any premature context switches, or even a high 
number of optimal context switches are detrimental to performance. Inter-process
communication (IPC) comes in a lot of flavors; files, sockets shared memory,
etc. all with varying overhead in addition to context switches between the
processes. With a thread-pool-driven solution, the host itself can schedule
their own plugins and communicate with minimal overhead.

In this thesis we suggest a compiler-pass-based approach to sandboxing these
plugins, and mitigating most problems. We aim to describe the problems as broad
as possible to not limit what the scope of the thesis more than necessary. Rack
Extensions, for instance, are written in C++, but the suggested solution is
intended to cover languages which can be compiled to a the LLVM intermediate
format in general. Specifically, the compiler pass described works on the LLVM
intermediate format itself and is unaware of the language which it has been 
compiled from. Rack Extensions are made by developers who are trusted not to
introduce anything malicious on purpose, but for the sake of this thesis, we
make sure to cover sandboxing of purposely malicious code to the best of our
abilities.
