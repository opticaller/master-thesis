\chapter {Sandboxing}

In order to limit what an extension or another piece of program can do,
restrictions must be imposed on code actually running on a system. Programs in
general should not interfere with each other's execution. Neither should they be
able to read or write each other's memory. It is also common to restrict which
processes are able to make system calls and restrict access to other resources,
such as files etc. Isolation of processes, enforcement of file-system
permissions, restricting resources to users with superuser privileges etc. are
responsibilities of the kernel with help from hardware, such as the
memory-management unit (MMU).

Ideally the same principles should apply for extensions as well. A program is
generally trusted to run code with its user's privileges, but extensions are
expected to only access to the resources they require to perform their specific
task. Ideally an extension would not inherit the privileges from its hosting
environment, but rather only be able to access resources which it's been
permitted to use.


\section {Memory Accesses}

A program should in general not be able to read from or write to another
program's memory. Not properly protecting programs can, for instance, result in
sensitive data, such as cryptographic keys or passwords in memory leaking.
Writing to another process' memory, either on purpose or accidentally, can
invalidate its executing state completely. On most modern systems this is taken
care of by memory paging, which triggers exceptions when a process tries to use
memory not assigned to it.

The same principles should arguably apply to plugins as well. A plugin should
not be allowed to peek or modify its host's data. Ideally, neither should it be
able to read from or write to other plugin instances. This relates to running
plugins in separate sandboxes per-instance, per-plugin class or using a single
shared one. In the context of process-based plugins, this is generally done by
the operating system itself. Thread-driven solutions however naturally share
memory, and the problem of isolating memory in them will be addressed later in
Chapter [ref-mem?]. We are not aware of previous research which addresses this
problem, as when desiring this type of memory protection, process isolation is
both the normal and natural solution.


\section {Restricting System Calls}

The ability of doing any input/output operations recides under normal
circumstances in system calls, an essential interface for requesting services
between a process and the operating system's kernel. Note that this is not
strictly the case, there exists embedded systems where programs are not
prevented from accessing the underlying hardware at all, but for most modern
systems, this is the case. Without the ability to perform any input/output
operations (any communication, not just `to screen', or `to file'), a process
would be completely isolated. Any work performed by the process would be local
to itself, and no external entity would observe it.

Were all system calls available to plugins however, they would essentially
have the same capabilities as the user running them. This would allow for
arbitrary code execution with the user's privileges. Many programs only need to
access a specific subset of these. For many, input/output to the standard
streams \texttt{stdin}, \texttt{stdout} and \texttt{stderr} are sufficient apart
from memory allocation. A controlled environment of execution can explicitly
reject a subset of system calls. A sandboxing environment can also, instead of
rejecting a program which uses them, emulate some of them, but in a harmless
manner. A program running under \texttt{chroot} for instance, has a different
directory as a root directory and can only access files in under it.

Depending on the sandboxing method used, restrictions to system calls might
require restricting access to methods which may perform system calls.
\texttt{printf} for instance, will perform a \texttt{write} system call and
\texttt{fopen} will perform an \texttt{open} system call. If only explicit
system calls are rejected, implicit ones can be used to circumvent many of them.
Restricting access to functions without side effect outside the program running
them, such as sorting, binary search and math functions is pointless however.
They are often useful and have no harmful side effects outside the program, so
long as memory isn't shared between processes.

In the context of plugins, which extend capabilities by performing a small set
of tasks, this is seldom required. Communicating back to the host, and not the
operating system directly is also often sufficient. Permitting arbitrary system
calls also means that the user must trust every plugin and every program they
would install and run equally.

Rack Extensions for instance operate on audio buffers in-memory, and don't need
to read/write files or raise signals on their own. Therefore a Rack Extension is
compiled against a stripped-down, slightly modified C standard library and API
functions specific to interacting with the plugin host. In addition to the
security aspect, a limited API also communicate how plugins should be written
and how they should behave.


\section {Existing Sandboxing Techniques}

Several sandboxing techniques exist with different purposes. Hardware
virtualization abstracts hardware and emulates a computer to enable an entire
operating-system environment to run inside it. Application virtualization on the
other hand encapsulates the program in a virtual environment, so when the
program believes that it's interfacing with the underlying operating system
directly, it's interfacing with the virtualized environment (which may, in turn
interface with the operating system). Though application virtualization may be
relevant, hardware virtualization is definitely outside the scope of this
thesis. All of the following sandbox techniques sandbox on a per-process level.

\subsection {libdetox}

...

\subsection {Chromium}

Unlike monolithic browsers, which run in a single protection domain, Chromium
is split into modules, the kernel and rendering engine. The browser kernel is
trusted and acts on behalf of the user, while the rendering engine is untrusted
and acts on behalf of ``the web'' with lowered privileges[ref-chrome].
Inherently, an attacker that is able to, for instance, exploit a vulnerability
in the HTML parser, the Document-Object Model (DOM) or the JavaScript
interpreter, would still be limited to the privileges of the rendering engine.

The architecture of Chromium isn't intended to prevent all kinds of attacks. A
compromised rendering engine could still attack other web sites. It's intended
to restrict an attacker (and the rendering engine itself) to using only the
browser kernel interface. As such it aims to protect the user's file system, to
prevent data loss or a ``drive-by malware installation'' for instance. The
architecture achieves fault tolerance by running a separate instance of the
rendering engine for each tab displaying content from the web.

On Windows Chromium relies on Windows-specific features to sandbox the rendering
engine. Before starting to render web content, the rendering engine irrevocably
lowers its own privileges. Instead of running with the user's Windows security,
the rendering engine runs with a greatly restricted security token.

By default, plug-ins for Chromium do not run in its sandbox however, as
plug-ins can have arbitrary behavior, some require accessing the filesystem,
others require network connections. They do run in a different process and
communicate with Chromium's kernel.

\subsection {Native Client}

Native Client (NaCl) is a sandboxing technology for running a subset of x86/ARM
native code safely using software-based fault isolation[ref-nacl]. Although
intended for browser-portable sandboxed binary modules, the research behind
Native Client is more generally applicable. Its purpose is to bring near-native
performance to running untrusted native code from a web browser with comparable
safety to JavaScript for instance.

For the purpose of building NaCl Modules a modified version of the GNU tool
chain have been used, including GCC,. A modified version of the
C-standard-library implementation newlib[ref-newlib] have been compiled against
the resulting toolchain, rehosted to use NaCl trampolines to implement system
services.

When presented to an user, an untrusted module may contain arbitrary code and
data. Therefore the NaCl runtime verifies that the module conforms to some
validity rules regarding, for instance, certain instructions, interrupts, system
calls. The runtime further uses segmented memory to constrain data and
instruction memory references. By using segmented memory, NaCl is able to
restrict access to shared libraries and other resources which might have been
deliberately mapped into the the address space of all processes by the operating
system, as, according to the authors, occur on the Windows platform.

Since the original paper related research has been done on extending Native
Client to language independance, including just-in-time compilation
and self-modifying code[ref-lang-indep] as well as Portable Native Client
(PNaCl) Executables[ref-pnacl] using the LLVM bitcode format as an intermediate
format to represent an architecture-agnostic portable executable which is
compiled in the runtime environment to the actual instruction-set architecture
(ISA). Native Client has been supported through the Chrome Web Store in Google
Chrome since version 14[ref-rel-14], released in late 2011. Native Client is
currently scheduled to be enabled for all pages as well following the release of
Portable Native Client[ref-pnacl-release].

<rel-14: \url{http://chrome.blogspot.se/2011/08/building-better-web-apps-with-new.html}>

<ref-pnacl-release: \url{https://developers.google.com/native-client/devguide/distributing#CWS}>
