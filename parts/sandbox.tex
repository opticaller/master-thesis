\chapter{Sandbox}

To limit what an extension or another piece of program can do, restrictions
must be imposed on code actually running on a system. Programs in general
should not interfere with each other's execution. Neither should they be able
to read each other's data. It is also common to restrict which processes are
able to make system calls and restrict access to other resources, such as files
etc. Enforcing file-system permissions, superuser privileges etc. are
responsibilities of the kernel.

The same principles apply for extensions. A main application is trusted to run
code with its user's privileges, but program extensions are expected to only
access to the resources they require to perform their specific task.

\section{Memory Accesses}

A program should in general not be able to read from or write to another
program's memory. Not properly protecting programs can for instance result in
sensitive data, such as cryptographic keys or passwords in memory leaking to
other processes. Writing to another process' memory, either on purpose or
accidentally, invalidate its executing state completely.

The same principles should apply to extensions as well. An extension should not
be allowed to peek or modify its host's data. Ideally, neither should it be
able to do the same for other program extensions.

\section{Restricting External Function Calls}

Restricting which library function calls are available to an extension limits
how it's able to communicate with the outside world. For example, restricting
access to stdin/stdout and other file descriptors, sockets or pipes.
