\chapter {Acronyms and Terms} \label{appA}

Acronyms and terms can be ambiguous in general, here follow definitions of how
they are used inside this thesis.

\begin {description}
	\item [API] Application Programming Interface, a specification intended to
		be used as an interface by software components to communicate with each
		other.
	\item [Context] The minimal set of data used by a task that must be saved
		to allow a task interruption, and a continuation of this task at the
		point it has been interrupted and at an arbitrary future date.
	\item [Context switching] The process of storing and restoring the
		\emph{context} of a \emph{CPU} so that execution can be resumed from
		the same point at a later time.
	\item [CPU] Central Processing Unit, the hardware within a computer system
		which carries out the instructions of a computer program.
	\item [Extension] A set of software components that adds specific abilities
		to a program.
	\item [GCC] GNU Compiler Collection, the standard compiler by most other
		modern Unix-like computer operating systems.
	\item [Host] Main program and environment in which one or more
		\emph{extensions} run.
	\item [IPC] See \emph{Inter-Process Communication}.
	\item [Instruction] A single operation of a processor defined by an
		instruction-set architecture.
	\item [Inter-Process Communication] A set of methods for the exchange of
		data among multiple threads in one or more processes.
	\item [Intrinsic function] A function available for use in a given language
		whose implementation is handled specially by the compiler.
	\item [Kernel] A program that constitutes the central core of a computer
		operating system. It has complete control over everything that occurs
		in the system.
	\item [Platform] A hardware architecture and a software framework
		(including application frameworks), where the combination allows
		software to run.
	\item [Process] An instance of a computer program that is being executed.
	\item [Signal] An asynchronous notification sent to a process or to a
		specific thread within the same process in order to notify it of an
		event that occurred. When a signal is sent, the operating system
		interrupts the target process's normal flow of execution.
	\item [SIMD] Single instruction, multiple data.
	\item [SIMD instruction] An instruction manipulating multiple data.
	\item [System call] A request for services from the operating system's
		kernel.
	\item [Thread] A lightweight \emph{process}. Multiple threads can exist
		within the same process and share resources such as memory, while
		different processes in general do not.
\end {description}

