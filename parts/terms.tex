\chapter{Acronyms and Terms}\label{appA}

Acronyms and terms can be ambiguous in general, here follow definitions of how
they are used inside this thesis.

\begin{description}
	\item[API] Application Programming Interface, a specification intended to
		be used as an interface by software components to communicate with each
		other.
	\item[Context] The minimal set of data used by a task that must be saved
		to allow a task interruption, and a continuation of this task at the
		point it has been interrupted and at an arbitrary future date.
	\item[Context switching] The process of storing and restoring the
		\emph{context} of a \emph{CPU} so that execution can be resumed from
		the same point at a later time.
	\item[CPU] Central Processing Unit, the hardware within a computer system
		which carries out the instructions of a computer program.
	\item[Extension] A set of software components that adds specific abilities
		to a program.
	\item[Host] Main program in which one or more \emph{extensions} run.
	\item[IPC] See \emph{Inter-Process Communication}.
	\item[Inter-Process Communication] A set of methods for the exchange of
		data among multiple threads in one or more processes.
	\item[Kernel]  A program that constitutes the central core of a computer
		operating system. It has complete control over everything that occurs
		in the system. 
	\item[Process] An instance of a computer program that is being executed.
	\item[System call] A request for services from the operating system's
		kernel.
	\item[Thread] A lightweight \emph{process}. Multiple threads can exist
		within the same process and share resources such as memory, while
		different processes in general do not.
\end{description}

